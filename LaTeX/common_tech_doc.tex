\documentclass[a4paper]{article}

\usepackage{fancyhdr} % not useful
\usepackage{listings}
\usepackage{xcolor}
\usepackage{graphicx} % \includegraphic{xxx.png}
\usepackage{times}
%\usepackage{indentfirst}
\usepackage[colorlinks, linkcolor=blue, pdftex,bookmarksnumbered]{hyperref}
\usepackage{blindtext} % test content
\usepackage{lipsum} % test content
\usepackage{titlesec} % define section style
\usepackage[margin=2cm]{geometry}

\pagestyle{fancy}
\renewcommand{\footrulewidth}{1pt}
\cfoot{Copyright \copyright\ 2017, Jinfeng Zhuang. All rights reserved.}
\rfoot{\thepage}

\setlength{\parindent}{0pt}

\let\oldsection\section
\renewcommand{\section}{\leftskip=2em \oldsection}
\let\oldsubsection\subsection
\renewcommand{\subsection}{\leftskip=2em \oldsubsection}
\let\oldsubsubsection\subsubsection
\renewcommand{\subsubsection}{\leftskip=2em \oldsubsubsection}

\definecolor{mygreen}{rgb}{0,0.6,0}
\definecolor{mygray}{rgb}{0.5,0.5,0.5}
\definecolor{mymauve}{rgb}{0.58,0,0.82}
\lstset{ %
backgroundcolor=\color{white},   % choose the background color
basicstyle=\footnotesize\ttfamily,        % size of fonts used for the code
columns=fullflexible,
breaklines=true,                 % automatic line breaking only at whitespace
captionpos=b,                    % sets the caption-position to bottom
tabsize=4,
commentstyle=\color{mygreen},    % comment style
escapeinside={\%*}{*)},          % if you want to add LaTeX within your code
keywordstyle=\color{blue},       % keyword style
stringstyle=\color{mymauve}\ttfamily,     % string literal style
frame=single,
rulesepcolor=\color{red!20!green!20!blue!20},
% identifierstyle=\color{red},
language=c++,
xleftmargin=3em
}

\title{Fingerprint on Cortex-M4}

\author{Jinfeng Zhuang}
\date{\today}

\begin{document}

\maketitle

\thispagestyle{empty}

\newpage

\begin{abstract}
To demostrate the product, we use Cortex-M4 to read data from ultrasonic fingureprint sample chip as a SPI master, then send processed data to PC on the USB-Device side.
\end{abstract}

\setcounter{page}{1}
\pagenumbering{Roman}

\newpage

\tableofcontents

\newpage

\setcounter{page}{1}
\pagenumbering{arabic}

\section{API}

\subsection{spi\_send}
\subsubsection{SYNOPSIS}
\begin{lstlisting}[language=C]
int spi_send(unsigned char *buffer, unsigned int length);
\end{lstlisting}
\subsubsection{DESCRIPTION}
Synchronous send bytes to SPI Slaver.
\subsubsection{PARAMETERS}
None.
\subsubsection{RETURN}
Return -1 if send failed.
\subsubsection{ERRORS}
None.

\subsection{spi\_recv}
\subsubsection{SYNOPSIS}
\begin{lstlisting}[language=C]
int spi_recv(unsigned char *buffer, unsigned int length);
\end{lstlisting}
\subsubsection{DESCRIPTION}
\subsubsection{PARAMETERS}
None.
\subsubsection{RETURN}
Return -1 if send failed.\\
Return 0 if there is no data.
\subsubsection{ERRORS}
None.

\subsection{usb\_send}
\subsubsection{SYNOPSIS}
\begin{lstlisting}[language=C]
int usb_send(unsigned char *buffer, unsigned int length);
\end{lstlisting}
\subsubsection{DESCRIPTION}
Synchronous send bytes to USB Host.
\subsubsection{PARAMETERS}
None.
\subsubsection{RETURN}
Return -1 if send failed.
\subsubsection{ERRORS}
None.

\subsection{usb\_recv}
\subsubsection{SYNOPSIS}
\begin{lstlisting}[language=C]
int usb_recv(unsigned char *buffer, unsigned int length);
\end{lstlisting}
\subsubsection{DESCRIPTION}

Asynchronous receive data in global buffer, which will be filled by USB interrupt handler.
\subsubsection{PARAMETERS}
None.
\subsubsection{RETURN}
Return -1 if recv failed.\\
Return 0 if there is no data.
\subsubsection{ERRORS}
None.

\section{Source Directory}

\end{document}

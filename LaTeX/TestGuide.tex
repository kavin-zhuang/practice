%
% For rede release report
%

\documentclass[a4paper]{article}

\usepackage{fancyhdr} % not useful
\usepackage{listings}
\usepackage{xcolor}
\usepackage{graphicx} % \includegraphic{xxx.png}
\usepackage{times}
%\usepackage{indentfirst}
\usepackage[colorlinks, linkcolor=blue, anchorcolor=blue, citecolor=blue, pdftex,bookmarksnumbered]{hyperref}
\usepackage{blindtext} % test content
\usepackage{lipsum} % test content
\usepackage{titlesec} % define section style
\usepackage[margin=2cm]{geometry}
\usepackage{enumerate}
\usepackage{mdwlist}

\newtheorem{remark}{Remark}

\pagestyle{fancy}
\renewcommand{\footrulewidth}{1pt}
\cfoot{\emph{Copyright \copyright\ 2017, NTESEC. All rights reserved.}}
\rfoot{\thepage}

\setlength{\parindent}{0pt}

\let\oldsection\section
\renewcommand{\section}{\leftskip=2em \oldsection}
\let\oldsubsection\subsection
\renewcommand{\subsection}{\leftskip=2em \oldsubsection}
\let\oldsubsubsection\subsubsection
\renewcommand{\subsubsection}{\leftskip=2em \oldsubsubsection}

\definecolor{mygreen}{rgb}{0,0.6,0}
\definecolor{mygray}{rgb}{0.5,0.5,0.5}
\definecolor{mymauve}{rgb}{0.58,0,0.82}
\lstset{ %
backgroundcolor=\color{white},   % choose the background color
basicstyle=\footnotesize\ttfamily,        % size of fonts used for the code
columns=fullflexible,
breaklines=true,                 % automatic line breaking only at whitespace
captionpos=b,                    % sets the caption-position to bottom
tabsize=2,
commentstyle=\color{mygreen},    % comment style
escapeinside={\%*}{*)},          % if you want to add LaTeX within your code
keywordstyle=\color{blue},       % keyword style
stringstyle=\color{mymauve}\ttfamily,     % string literal style
frame=single,
rulesepcolor=\color{red!20!green!20!blue!20},
% identifierstyle=\color{red},
language=c++,
xleftmargin=3em
}

%\includegraphics[scale=0.2]{rede.png} 

\title{Reworks Test Guide}

\author{Jinfeng Zhuang \\ \\ jinfeng.zhuang@aliyun.com}
\date{}

%=====================================================================================

\begin{document}

\maketitle

\vspace{130pt}

\begin{center}
%\includegraphics[scale=0.5]{timthumb.jpg}
\end{center}

\thispagestyle{empty}

%=====================================================================================
\newpage
\setcounter{page}{1}
\pagenumbering{Roman}

\tableofcontents

%=====================================================================================
\newpage
\setcounter{page}{1}
\pagenumbering{arabic}

\section{Introduction}

All the testpoint is a measure of data transfer, and they are based on high-precision timer.\\
Test object access can be directly mode or user interface mode. For different developers.\\
Test benifite:

support demos of how to use the application interface.

support a model to how to write a model release report.

%=====================================================================================

\section{What-Why-How}

Test is a method for user to learn the supported stack and the interface. A large software architecture is divided into different module on different layers, each module should be a test point, and the developer of the module has the responsibility test it's dependent modules, and support a test report to it's users, which we called the theoretical value.

Each module test include function test and time token test, and heap, stack allocation test, may have a pressure test.

The contents are orgnized from different source, such as vxWorks test benches.

If do performance test, you should disable the interrupt.

While do test, u can only use the related system resource, and put the final result to console.

Below is a simple diagram for embeded system architecture:\\

\begin{tabular}{|c|c|c|}
\hline \multicolumn{3}{|c|}{Product}\\
\hline Socket & File System & Graphics\\
\hline SymbolTable & SHELL & Log\\
\hline\multicolumn{3}{|c|}{POSIX}\\
\hline\multicolumn{3}{|c|}{OS}\\
\hline\multicolumn{2}{|c|}{BSP} & CSP\\
\hline\multicolumn{3}{|c|}{C Library}\\
\hline
\end{tabular}

%=====================================================================================
\newpage
\section{Code Rules}

This section describe how to write a test program, how to get the result automatively. The destination of test should be automative. After machine auto run, the result will give with a paper.

A high-pricision timer is required to measure the interfaces' time-token.

%=====================================================================================
\newpage
\section{Collection}

\subsection{Realtime Performance}
\subsection{Time}

%=====================================================================================
\newpage
\section{C Library}

C library is supported by tool chain, which is the basic mechisiam for embeded system, more basic is assembly instructions.

\subsection{common}



\subsection{IO}


\subsection{string}


\subsubsection{strcpy}


\subsection{wchar}


\subsection{math}


\subsection{signal}


\subsection{locale}


\subsection{varlist}


\subsection{time}


\subsection{syscall}


\subsection{setjmp}


\subsection{hash}



\subsection{memory}


\subsubsection{memalign}


\subsubsection{memmove}


\subsubsection{memset}

%=====================================================================================
\newpage
\section{Core Supported Package}

\subsection{High-Pricision Timer}
\begin{tabular}{|c|c|c|}
\hline Interval\\
\hline Pricision\\
\hline CPU freq(Delay with I-Cache)\\
\hline Overflow Handler\\
\hline Interrupt Execution Time\\
\hline tickAnnounce\\
\hline
\end{tabular}
\subsection{Clock}
get and set system clock.
\subsection{Cache}


\subsection{MMU}


\subsection{Float}


\subsection{CPU Frequence}


\subsection{Barriar}



%=====================================================================================
\newpage
\section{Board Supported Package}

Include I/O interface and block interface.

\subsection{Flash Directly Access Baudrate}


\subsection{Serial ioctl interface and IO test}


\subsection{RTC}

%=====================================================================================
\newpage
\section{Operation System}

\subsection{Synchrone Mechanism}
x

\subsubsection{semaphore}
x

\subsection{Multicore Synchrone Mechanism}

x
\subsubsection{spinlock}
x

\subsubsection{semaphore}
x

\subsection{Task}
x

\subsubsection{task switch}
x

\subsubsection{task switch time taken}
x

\subsubsection{task switch between multicores}

x
\subsubsection{task sequence control}
x

\subsubsection{task execution time \& sleep time}

x
\subsection{Task Sequence Control}
x

\subsection{Exception Response}
x


\subsubsection{Limit of Interrut Lock}

x
\subsection{System Call}
x

\subsubsection{Sleep}

x
\subsubsection{Software Watchdog time token}

x

%=====================================================================================
\newpage
\section{POSIX}
x


\subsection{Timer}
x

\subsection{Semaphore}
x

\subsection{Thread}

x
\subsection{rwlock}

x
\subsection{Mutex}
x

\subsection{Message Queue}

x
\subsection{var}

x
\subsection{key}

x
\subsection{cond}
x

\subsection{mpart}
x

\subsection{watchdog}

x

%=====================================================================================
\newpage
\section{vxWorks Compatible Layer}
x

\subsection{float benchmark}
x

\subsection{heap}

x
\subsection{message queue}
x

\subsection{semaphore B/C/M performance}
x


\subsection{task}
x

\subsubsection{manager}

x

priority reserve test is a very important point.

\subsubsection{hook}
x

\subsubsection{var}

x
\subsubsection{info}
x


\subsection{pipe}
x
\subsubsection{create}

x
\subsubsection{delete}

x
\subsubsection{read/write}
x


\subsection{watchdog}

x

\subsection{clock}
x

\subsubsection{get}
x

\subsubsection{set}
x

\subsubsection{notify}
x


\subsection{timer}
x

\subsubsection{connect}
x

\subsubsection{cancel}

x

\subsection{interrupt}
x

\subsubsection{context}

x
\subsubsection{nest}

x
\subsubsection{relation}
x


\subsection{memory}

x
\subsubsection{memory device operation}

x
\subsubsection{interface}
x

\subsubsection{partition}
x

\subsubsection{manager}
x

\subsubsection{mem pool}
x

\subsubsection{mem share}

x

\subsection{errno}
x

\subsubsection{get}

x
\subsubsection{set}
x


\subsection{common}

x
\subsubsection{task event}

x
\subsubsection{semaphore}
x

\subsubsection{message queue}
x


\subsection{watchdog time token}

x
\subsection{Synchrone}

x
\subsection{ CPU}

x
\subsubsection{ CPU affinity}
x

\subsubsection{ scheduler on SMP}

x
\subsubsection{ CPU Lock}
x

\subsubsection{ CPU On/Off}
x



%=====================================================================================
\newpage
\section{File System}

x

\subsection{fread, fwrite on different storage media}
x


%=====================================================================================
\newpage
\section{Socket}

x

\subsection{TCP}

x
\subsection{UDP}
x

\subsection{Broadcast}

x
\subsection{Multicast}

x
\subsection{select}
x

\subsection{route}
x

\subsection{ARP}
x

\subsection{ICMP: Ping with Big Package}

x

%=====================================================================================
\newpage
\section{Graphics}

see SVN test folder.

%=====================================================================================
\newpage
\section{Product}

x

\subsection{boot time}

x
\subsection{gtk dma}
x

\subsection{monitor: cpuuse}

x
\subsection{fsinfo}
x

\subsection{meminfo}
x

\subsection{memory leak test}

x
\subsection{task meminfo}

x
\subsection{sem stat}

x
\subsection{stackinfo}

x

%=====================================================================================
\newpage
\section{Reference}


EEMBC: it support a bench test and reference data.\\
IEEE\\
POSIX\\
MISRC-2004\\

\end{document}
